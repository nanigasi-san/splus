
\documentclass[11pt,a4j]{jreport}

\usepackage{comment}
\usepackage{float}
\usepackage{color}
\usepackage{multicol}
\usepackage[dvipdfmx]{pict2e}
\usepackage{wrapfig}
\usepackage{graphicx}
\usepackage{bm}
\usepackage{url}
\usepackage{underscore}
\usepackage{colortbl}
\usepackage{tabularx}
\usepackage{fancyhdr}
\usepackage{ulem}
\usepackage{cite}
\usepackage{amsmath,amssymb,amsfonts}
\usepackage{algorithmic}
\usepackage{textcomp}
\usepackage{xcolor}
\usepackage[ipaex]{pxchfon}


\usepackage[top=30truemm,bottom=30truemm,left=25truemm,right=25truemm]{geometry}

\begin{document}

\thispagestyle{empty}
\begin{center}

\vspace{40mm}
{\huge\noindent\textbf{KEG上におけるクロスキャップ数の計算}}\\
\vspace{20mm}

{\Large\noindent
2022年10月21日\\
\vspace{\baselineskip}
Kaito Yamada\\
}
% \vspace{40mm}


\end{center}

\thispagestyle{empty}
\clearpage

%=====================================================================================

\begin{abstract}
    研究の要旨。なんやかんやなんやかんやなんやかんやなんやかんやなんやかんやなんやかんやなんやかんやなんやかんやなんやかんやなんやかんやなんやかんやなんやかんやなんやかんやなんやかんやなんやかんやなんやかんやなんやかんやなんやかんやなんやかんやなんやかんや
\end{abstract}

%=====================================================================================

% 目次の表示
\tableofcontents

%=====================================================================================
\pagestyle{fancy}
\lhead{\rightmark}
\renewcommand{\chaptermark}[1]{\markboth{第\ \normalfont\thechapter\ 章~~#1}{}}


%1章
\chapter{はじめに} %章

\section{研究背景} %1.1
\subsection{S${}^\text{+}$のアルゴリズムとは} %1.1.1
\subsection{KEGとは} %1.1.2


\section{先行研究}
\subsection{C(K)=3の場合}
\subsection{KEGの同値判定}

\section{研究目的}

%2章
\chapter{アルゴリズムと計算量}

\section{S${}^\text{+}$のアルゴリズム}
\subsection{KEGにおけるS${}^\text{+}$のアルゴリズム}
\subsection{操作例}
\subsection{計算量}

\section{グラフの拡張}
\subsection{頂点の分割}
\subsection{RI${}^\text{+}$のアルゴリズム}
\subsection{計算量}

\chapter{まとめ}
研究のまとめ。なんやかんやなんやかんやなんやかんやなんやかんやなんやかんやなんやかんやなんやかんやなんやかんやなんやかんやなんやかんやなんやかんやなんやかんやなんやかんやなんやかんやなんやかんやなんやかんやなんやかんやなんやかんやなんやかんやなんやかんや

%=====================================================================================
\chapter*{謝辞} %章を付けずにタイトル表示
\addcontentsline{toc}{chapter}{謝辞} %章立てせずに目次に追加するおまじない
本論文を作成するにあたり、---- みなさまに感謝の意を表します.

%=====================================================================================

\addcontentsline{toc}{chapter}{参考文献} %章立てせずに目次に追加するおまじない
\renewcommand{\bibname}{参考文献} %これがないと,タイトルが「関連図書」になってしまう
\bibliography{paper.bib} %bibtexファイルの読み込み
\bibliographystyle{junsrt} %本文に\cite{}を入れることで,参考文献表示
\cite{keg}
\cite{ck3}
\end{document}
